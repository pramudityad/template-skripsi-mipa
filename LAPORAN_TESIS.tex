% -----------------------------------------------------------------------
% Template Skripsi untuk MIPA
% 
% @author Yusuf Syaifudin
% @created 29/02/2016
% 
% -----------------------------------------------------------------------

\documentclass[ugmtesis]{ugmtesis}

\input{ADDITIONAL_PACKAGES}

% Konfigurasi variable seperti judul dan lain sebagainya
\titleind{AUTENTIKASI MESIN KE MESIN BERBASIS RISIKO PADA KASUS \textit{FAST HEALTH INTEROPERABILITY RESOURCES} MENGGUNAKAN RANDOM FOREST}

\titleeng{RISK BASED MACHINE TO MACHINE AUTHENTICATION IN \textit{FAST HEALTH INTEROPERABILITY RESOURCES} CASE USING RANDOM FOREST}

\fullname{Damar Arba Pramuditya}

\idnum{22/501365/PPA/06386}

\examdate{17 Januari 2024}

\degree{Master of Computer Science}

\yearsubmit{2024}

\program{Magister Ilmu Komputer}

\headprogram{Aina Musdholifah, S.Kom., M.Kom. Ph.D}

\dept{Ilmu Komputer dan Elektronika}

\firstsupervisor{Dr. Lukman Heryawan, S.T., M.T.}

\firstexaminer{Wahyono, S.Kom., Ph.D.}

\secondexaminer{Aina Musdholifah, S.Kom., M.Kom. Ph.D}

\thirdexaminer{Dr. Agus Sihabuddin, S.Si., M.Kom.}



\begin{document}

%-----------------------------------------------------------------
% Disini awal masukan untuk muka skripsi
%-----------------------------------------------------------------

% Cover
\cover

% Halaman judul
\titlepageind 

% Halaman Persetujuan
\approvalpage

% Halaman Pernyataan
\declarepage

% Halaman Persembahan
\acknowledment
\input{BAB_TESIS/BAB0/1_PERSEMBAHAN}

%-----------------------------------------------------------------
% Disini akhir masukan untuk muka skripsi
%-----------------------------------------------------------------

% Motto
\input{BAB_TESIS/BAB0/2_MOTTO}

% Prakata
\input{BAB_TESIS/BAB0/3_PRAKATA}

%-----------------------------------------------------------------
% Daftar Isi
%-----------------------------------------------------------------
\newpage
\phantomsection
\addcontentsline{toc}{chapter}{\contentsname}
\tableofcontents
%-----------------------------------------------------------------
% Akhir Daftar Isi
%-----------------------------------------------------------------

%-----------------------------------------------------------------
% Daftar Tabel
%-----------------------------------------------------------------
\newpage
\phantomsection
\addcontentsline{toc}{chapter}{\listtablename}
\listoftables
%-----------------------------------------------------------------
% Akhir Daftar Tabel
%-----------------------------------------------------------------

%-----------------------------------------------------------------
% Daftar Gambar
%-----------------------------------------------------------------
\newpage
\phantomsection
\addcontentsline{toc}{chapter}{\listfigurename}
\listoffigures
%-----------------------------------------------------------------
% Akhir Daftar Gambar
%-----------------------------------------------------------------


%-----------------------------------------------------------------
%Disini awal masukan Intisari
%-----------------------------------------------------------------
\begin{abstractind}
	Studi ini menggunakan pendekatan berbasis risiko untuk mengidentifikasi dan menilai potensi risiko yang terkait dengan otentikasi Machine-to-Machine (M2M). Hal ini melibatkan analisis pelaku ancaman potensial, kerentanan, dan dampak serangan yang berhasil. Studi ini juga mengevaluasi metode otentikasi M2M saat ini dan efektivitasnya dalam memitigasi risiko yang teridentifikasi, menggunakan algoritma Random Forest untuk mengklasifikasikan akses. Selain itu, penelitian ini mengusulkan strategi untuk meningkatkan otentikasi M2M guna mengurangi risiko serangan yang berhasil.

Temuan ini diharapkan dapat menyoroti beberapa risiko yang terkait dengan otentikasi M2M, seperti akses perangkat yang tidak sah, di mana peretas mengeksploitasi kerentanan untuk mencuri informasi sensitif, dan serangan penolakan layanan (DoS), yang mengganggu komunikasi M2M dan menyebabkan waktu henti sistem. Dalam konteks Fast Healthcare Interoperability Resources (FHIR), hal ini penting karena FHIR digunakan untuk bertukar data kesehatan elektronik antara sistem dan perangkat medis. Penelitian ini mencapai akurasi 0,708, presisi 0,701, recall 0,968, dan skor F1 0,813 dengan pengklasifikasi Random Forest.

Studi ini memberikan wawasan berharga tentang risiko yang terkait dengan otentikasi M2M dan menawarkan strategi mitigasi. Temuan penelitian ini sangat berguna bagi organisasi yang menerapkan perangkat IoT, khususnya di sektor teknologi perawatan kesehatan. Dengan mengatasi kerentanan dan mengusulkan metode autentikasi yang efektif, penelitian ini bertujuan untuk secara signifikan mengurangi risiko terkait autentikasi M2M di lingkungan sensitif seperti layanan kesehatan.
\end{abstractind}
%-----------------------------------------------------------------
%Disini akhir masukan Intisari
%-----------------------------------------------------------------

%-----------------------------------------------------------------
%Disini awal masukan untuk Abstract
%-----------------------------------------------------------------
\begin{abstracteng}
  This study employs a risk-based approach to identify and assess potential risks associated with Machine-to-Machine (M2M) authentication. Threat actor analysis, vulnerability assessment, and evaluation of the impact of attacks are conducted, along with an evaluation of current M2M authentication methods. The research develops strategies to enhance authentication to mitigate attack risks. The increase in IoT devices in healthcare technology emphasizes the importance of secure inter-device authentication. Various approaches such as encryption models and authentication protocols have been proposed, but they fall short of a risk-based approach that integrates threat analysis with the evaluation of authentication effectiveness.

This study utilizes the Random Forest algorithm to classify access and assess the effectiveness of current authentication methods. Key findings include the identification of risks such as unauthorized device access like replay attacks. In the context of Fast Healthcare Interoperability Resources (FHIR), this study achieves an accuracy of 0.708, a precision of 0.701, a recall of 0.968, and an F1 score of 0.813 using the Random Forest algorithm.

For comparison, a study using the Local Outlier Factor (LOF) for swipe-based user authentication on smartphones showed that even without optimization, the model could achieve success rates of over 90\% with a FAR of up to 40\%. This indicates that LOF can provide competitive authentication metrics with complex models like Random Forest.

This research offers valuable insights for organizations implementing IoT devices, particularly in the healthcare technology sector, to reduce risks associated with M2M authentication.

Keywords: Machine-to-Machine (M2M) Authentication, Fast Healthcare Interoperability Resources (FHIR), Random Forest, Local Outlier Factor (LOF), Authentication Improvement
\end{abstracteng}
%-----------------------------------------------------------------
%Disini akhir masukan Abstract
%-----------------------------------------------------------------


%-----------------------------------------------------------------
% Awal BAB 1
%-----------------------------------------------------------------
\chapter{PENDAHULUAN}
\label{PENDAHULUAN}

	\section{Latar Belakang}
	\label{pendahuluan latar belakang}
	Risk-based M2M (Machine-to-Machine) authentication merupakan metode otentikasi yang mengukur tingkat risiko yang terkait dengan suatu perangkat atau sistem dan menyesuaikan tingkat otentikasi yang diperlukan sesuai dengan tingkat risiko tersebut. Dalam sistem kesehatan, FHIR (Fast Healthcare Interoperability Resources) menjadi standar yang digunakan untuk pertukaran informasi kesehatan secara elektronik. Kerangka kerja OAuth menyediakan autentikasi dan otorisasi menggunakan profil dan kredensial pengguna di penyedia identitas yang ada. Hal ini membuat memungkinkan penyerang untuk mengeksploitasi kerentanan apa pun yang timbul dari pertukaran data dengan penyedia. Kerentanan dalam OAuth Alur otorisasi OAuth memungkinkan penyerang untuk mengubah urutan alur normal protokol OAuth (Rahat, Tamjid Al et al., 2021). Sehingga, sistem otentikasi FHIR saat ini hanya didasarkan pada OAuth2 dan OpenID Connect, sehingga risiko dari perangkat yang terhubung tidak diperhitungkan dalam otentikasi.

FHIR adalah sebuah acuan / standar yang digunakan dalam pertukaran informasi tentang kesehatan secara elektronik atau online. FHIR dikembangkan dan diawasi oleh sebuah organisasi yang bernama HL7 (Health Level Seven International) (Mark L. Braunstein, 2022) . HL7 adalah sebuah non-profit organisasi yang menyediakan sebuah framework dan acuan-acuan dalam pertukaran, integrasi, pembagian dan penerimaan informasi tentang kesehatan yang dapat membantu praktik dalam kesehatan, manajemen serta evaluasi pelayanan kesehatan. Dalam konteks FHIR, ini penting karena FHIR digunakan untuk pertukaran data kesehatan elektronik antar sistem dan perangkat medis (Solapurkar, 2016). Karena FHIR digunakan untuk mengakses data kesehatan yang sensitif, penting untuk memastikan bahwa hanya perangkat dan sistem yang sah yang diizinkan untuk mengakses data. Namun, tidak semua perangkat atau sistem memiliki tingkat risiko yang sama (Dutson et al., 2019). Misalnya, perangkat medis yang digunakan untuk mengadministrasikan obat kepada pasien memiliki risiko yang lebih tinggi dibandingkan dengan sensor suhu di ruangan.

Salah satu serangan yang umum terjadi pada kasus autentikasi token ini adalah replay attack , bentuk serangan jaringan di mana transmisi data yang valid diulang atau ditunda secara jahat atau curang. Dengan mengimplementasikan metode autentikasi berbasis risiko (Stephan Wiefling et al., 2021), sistem dapat menyesuaikan tingkat keamanan yang dibutuhkan sesuai dengan tingkat risiko dari perangkat atau sistem yang berkomunikasi, sehingga dapat meningkatkan keamanan dalam pertukaran data kesehatan melalui FHIR.

	\section{Rumusan Masalah}
	\label{pendahuluan rumusan masalah}
	\begin{enumerate}
    \item FHIR masih bergantung pada external identity management sistem untuk otentikasi dan otorisasi.
    \item FHIR tidak memiliki mekanisme otentikasi yang mempertimbangkan risiko dari perangkat yang terhubung.
    \item Masih menggunakan \textit{singe factor authentication} yang rentan terhadap serangan \textit{token replay}.
\end{enumerate}

	\section{Batasan Masalah}
	\label{pendahuluan batasan masalah}
	Agar penelitian ini dapat dilakukan dengan baik, maka perlu dibuat batasan masalah. Batasan masalah pada penelitian ini adalah:

\begin{enumerate}
    \item Penelitian ini hanya akan memfokuskan pada risiko yang terkait dengan otentikasi M2M pada FHIR.
    \item Datasek yang digunakan dalam penelitian ini adalah data sekunder yang diperoleh dari literatur yang relevan.
    \item Pemilihan fitur yang digunakan dalam penelitian ini adalah fitur yang relevan dengan risiko otentikasi M2M pada FHIR.
\end{enumerate}

	\section{Tujuan Penelitian}
	\label{pendahuluan tujuan penelitian}
	Tujuan penelitian ini adalah mengimplementasikan sistem autentikasi mesin ke mesin berbasis risiko yang dapat meningkatkan keamanan sistem autentikasi yang nantinya dapat digunakan dalam sistem penyedia layanan kesehatan.

	\section{Manfaat Penelitian}
	\label{pendahuluan manfaat penelitian}
	Manfaat penelitian yang didapat sebagai berikut:

\begin{enumerate}
    \item Dapat memodelkan masalah otentikasi mesin ke mesin berbasis risiko pada FHIR.
    \item Meminimalisir risiko yang terkait dengan otentikasi mesin ke mesin pada FHIR.
    \item Menganalisa apakah otentikasi mesin ke mesin berbasis risiko dengan Random Forest dapat meningkatkan keamanan sistem otentikasi pada FHIR.
\end{enumerate}

	% \section{Sistematika Penulisan}
	% \label{pendahuluan sistematika penulisan}
	% \input{BAB_TESIS/BAB1/6_SISTEMATIKA_PENULISAN}

%-----------------------------------------------------------------
% Akhir BAB 1
%-----------------------------------------------------------------


%-----------------------------------------------------------------
% Awal BAB 2
%-----------------------------------------------------------------
\chapter{TINJAUAN PUSTAKA}
\label{TINJAUAN PUSTAKA}
\input{BAB_TESIS/BAB2/1_TINJAUAN_PUSTAKA}

%-----------------------------------------------------------------
% Akhir BAB 2
%-----------------------------------------------------------------


%-----------------------------------------------------------------
% Awal BAB 3
%-----------------------------------------------------------------
\chapter{DASAR TEORI}
\label{DASAR TEORI}

	\section{FHIR \textit{(Fast Healthcare Interoperability Resources)}}
	\label{dasar teori fhir}
	FHIR, singkatan dari Fast Healthcare Interoperability Resources, merupakan standar internasional yang diperkenalkan oleh Health Level Seven International (HL7) untuk memfasilitasi pertukaran data kesehatan elektronik. Standar ini dirancang untuk mengatasi tantangan interoperabilitas antara sistem-sistem informasi kesehatan yang beragam, dengan tujuan memungkinkan pertukaran data yang cepat, fleksibel, dan terstandarisasi di seluruh industri kesehatan. 

FHIR menggunakan format data yang ringan seperti JSON atau XML, dan protokol komunikasi web standar seperti HTTP atau HTTPS, yang memfasilitasi integrasi dengan sistem-sistem modern dengan lebih mudah. Dengan pendekatan moduler, FHIR memungkinkan akses granular terhadap informasi kesehatan, sesuai kebutuhan aplikasi atau pengguna. 

Adopsi FHIR diharapkan dapat meningkatkan interoperabilitas di seluruh rantai perawatan kesehatan, memungkinkan pertukaran informasi yang lebih efisien dan akurat, serta mendukung pengembangan aplikasi kesehatan yang inovatif dan terintegrasi. Sebagai hasilnya, FHIR juga membuka pintu bagi pengembangan solusi-solusi teknologi kesehatan yang lebih canggih, seperti analisis big data dan kecerdasan buatan, serta integrasi dengan perangkat medis wearable. 

	\section{Autorisasi}
	\label{Autorisasi}
	Otorisasi merujuk pada proses yang menentukan hak akses yang diberikan kepada entitas setelah autentikasi identitasnya berhasil dilakukan. Otorisasi memainkan peran penting dalam mengatur akses ke sumber daya dan layanan di dalam suatu sistem. Ini melibatkan penentuan apakah subjek atau entitas memiliki izin yang sesuai untuk melakukan tindakan tertentu dalam lingkungan yang diberikan. Proses otorisasi sering kali dilakukan setelah proses autentikasi yang sukses, di mana autentikasi memverifikasi identitas entitas. Dengan adanya otorisasi, sistem dapat memastikan bahwa hanya entitas yang memiliki hak yang sesuai yang diberikan akses ke sumber daya atau layanan tertentu, yang pada gilirannya membantu menjaga keamanan sistem secara keseluruhan. Misalnya, dalam sebuah aplikasi perbankan, setelah seorang pengguna berhasil mengautentikasi identitasnya, proses otorisasi akan menentukan hak akses pengguna tersebut terhadap fungsi-fungsi seperti pengecekan saldo, transfer dana, atau pembayaran tagihan. Oleh karena itu, pemahaman yang mendalam tentang konsep otorisasi penting untuk merancang dan mengimplementasikan sistem informasi yang aman dan efektif.


	\section{Autentikasi}
	\label{Autentikasi}
	Autentikasi adalah konsep fundamental yang diperlukan untuk memvalidasi keaslian identitas entitas tertentu dalam suatu sistem. Identitas, sebagai inti dari autentikasi, merujuk pada informasi yang digunakan untuk mengidentifikasi subjek. Kredensial, sebagai elemen kunci dalam proses autentikasi, terdiri dari informasi otentikasi yang diperlukan untuk membuktikan identitas subjek, seperti kata sandi, token, atau biometrik.

Metode autentikasi beragam dan dapat mencakup kata sandi, token, biometrik, sertifikat digital, serta otorisasi multi-faktor (MFA). Protokol autentikasi, sebagai serangkaian langkah atau aturan, memberikan panduan bagi pelaksanaan autentikasi dalam suatu sistem, contohnya OAuth, OpenID, SAML, dan Kerberos.

Keamanan merupakan aspek krusial dalam autentikasi, yang mencakup kerahasiaan kredensial, integritas data autentikasi, dan non-repudiasi. Pemahaman akan kelemahan dan ancaman terhadap sistem autentikasi, seperti serangan phishing, brute force, dan man-in-the-middle, penting untuk meningkatkan ketahanan sistem.

Selain itu, autentikasi harus dapat diandalkan, sehingga sistem dapat memberikan verifikasi identitas yang konsisten dan akurat
		\subsection{Standar Autentikasi Pada FHIR}
		\label{standar autentikasi pada fhir}
		Hubungan antara autentikasi dan FHIR berkaitan dengan keamanan dan akses kontrol dalam pertukaran data kesehatan elektronik. Autentikasi digunakan untuk memverifikasi identitas entitas yang terlibat dalam pertukaran data menggunakan standar FHIR. Setelah identitas tersebut diverifikasi, otorisasi diterapkan untuk menentukan hak akses entitas tersebut terhadap data yang disediakan oleh layanan FHIR.

Dalam konteks FHIR, autentikasi digunakan untuk memastikan bahwa entitas yang mencoba mengakses atau menyediakan data kesehatan melalui API FHIR adalah entitas yang sah. Ini bisa berarti memverifikasi identitas pengguna, aplikasi, atau sistem yang berusaha berinteraksi dengan layanan FHIR. Autentikasi bisa dilakukan menggunakan berbagai metode, seperti kata sandi, token, atau mekanisme autentikasi yang lebih kuat seperti sertifikat digital atau biometrik, tergantung pada kebutuhan dan kebijakan keamanan sistem.

Setelah autentikasi berhasil dilakukan, otorisasi diterapkan untuk menentukan apa yang diizinkan entitas tersebut lakukan dengan data yang tersedia melalui layanan FHIR. Misalnya, seorang dokter mungkin memiliki akses penuh untuk melihat dan mengubah catatan medis pasien tertentu, sementara seorang petugas administrasi hanya diizinkan untuk melihat informasi dasar pasien tanpa memiliki kemampuan untuk mengubahnya. Otorisasi dalam konteks FHIR memastikan bahwa akses ke data kesehatan dikontrol sesuai dengan kebutuhan dan kebijakan privasi yang berlaku.

Dengan demikian, autentikasi dan otorisasi berperan penting dalam menjaga keamanan dan kerahasiaan data kesehatan yang ditangani oleh layanan FHIR, memastikan bahwa hanya entitas yang berwenang yang dapat mengakses informasi yang sensitif dan penting tersebut.


		\subsection{Autentikasi Mesin ke Mesin}
		\label{autentikasi mesin ke mesin}
		Machine-to-Machine (M2M) authentication adalah proses verifikasi yang digunakan untuk mengautentikasi perangkat atau mesin yang terhubung ke jaringan, seperti komputer, perangkat IoT, atau perangkat mobile. Proses ini memastikan bahwa hanya perangkat yang sah yang dapat terhubung ke jaringan dan mengakses data atau layanan yang tersedia seperti skema pada Gambar \ref*{fig:m2m}.

M2M authentication dapat menggunakan berbagai metode, seperti pengenalan suara, pengenalan wajah, pengenalan sidik jari, atau kombinasi dari metode tersebut. Dalam beberapa kasus, M2M authentication juga dapat menggunakan teknologi kriptografi, seperti enkripsi atau sertifikat digital, untuk memastikan keamanan komunikasi antar perangkat.

\begin{figure}
    \centering
    \includegraphics[width=0.8\textwidth]{BAB_TESIS/IMAGES/m2m_auth.png}
    \caption{Skema M2M Authentication}
    \label{fig:m2m}
\end{figure}

		\subsection{Metode Autentikasi Mesin ke Mesin}
		\label{metode autentikasi mesin ke mesin}
		Salah satu metode autentikasi Machine-to-Machine (M2M) menggunakan token merujuk pada proses verifikasi identitas antara dua atau lebih perangkat atau sistem tanpa intervensi manusia. Dalam skenario ini, token digunakan sebagai kredensial atau kunci otentikasi yang diberikan kepada perangkat atau sistem untuk membuktikan identitasnya kepada sistem yang lain.


			\subsubsection{\textit{Basic Access Authentication}}
			\label{basic access authentication}
			\input{BAB_TESIS/BAB3/3_3_1_BASIC_ACCESS_AUTHENTICATION}

			\subsubsection{\textit{Token}}
			\label{token}
			Klien membuat permintaan ke server otorisasi dengan mengirimkan ID klien, rahasia klien, bersama dengan audiens dan klaim-klaim lainnya. Server otorisasi memvalidasi permintaan tersebut, dan, jika berhasil, mengirimkan respons dengan token akses. Klien sekarang dapat menggunakan token akses untuk meminta sumber daya yang dilindungi dari server sumber daya.
Karena klien harus selalu menjaga rahasia klien, pemberian ini hanya dimaksudkan untuk digunakan pada klien terpercaya. Dengan kata lain, klien yang menyimpan rahasia klien harus selalu digunakan di tempat di mana tidak ada risiko rahasia tersebut disalahgunakan. Sebagai contoh, meskipun mungkin ide yang baik untuk menggunakan hibah kredensial klien di sistem internal yang mengirimkan laporan di seluruh web ke bagian lain dari sistem Anda, namun tidak dapat digunakan untuk alat publik yang dapat diakses oleh pengguna eksternal mana pun.
Berikut ini adalah permintaan HTTP yang relevan pada Tabel \ref{tab:req_http} berikut:

\begin{table}[H]
    \caption{Permintaan HTTP}
    \vspace{0.5em}
    \centering
    \begin{tabular}{|c|c|c|}
        \hline
        Permintaan & Deskripsi \\
        \hline \hline
        POST & Metode HTTP \\
        \hline
        /token & Endpoint \\
        \hline
        grant\_type=client\_credentials & Jenis hibah \\
        \hline
        & ID klien \\
        \hline
        & Rahasia klien \\
        \hline
        & Audiens \\
        \hline
    \end{tabular}
    \label{tab:req_http}
\end{table}

Sedangkan berikut contoh respon HTTP yang relevan pada Tabel \ref{tab:res_http} berikut:

\begin{table}[h]
    \caption{Respon HTTP}
    \vspace{0.5em}
    \centering
    \begin{tabular}{|c|c|c|}
        \hline
        Respon & Deskripsi \\
        \hline \hline
        200 OK & Kode status HTTP \\
        \hline
        Content-Type: application/json & Header HTTP \\
        \hline
        Cache-Control: no-store & Header HTTP \\
        \hline
        Pragma: no-cache & Header HTTP \\
        \hline
        \{ & Body \\
        \hline
        "access\_token": "2YotnFZFE & \\
        \hline
        "token\_type": "example", & \\
        \hline
        "expires\_in": 3600, & \\
        \hline
        "example\_parameter": "example\_value" & \\
        \hline
        \} & \\
        \hline
    \end{tabular}
    \label{tab:res_http}
\end{table}

	\section{\textit{Risk-Based Authentication}}
	\label{risk-based authentication}

	\section{\textit{Classification and Regression Tree (CART)}}
	\label{classification and regression tree}

		\subsection{\textit{Random Forest}}
		\label{random forest}


		\subsection{Laju Galat klasifikasi}
		\label{laju galat klasifikasi}

		\subsection{\textit{Variable Importance Measure(VIM)}}
		\label{variable importance measure}


	

%-----------------------------------------------------------------
% Akhir BAB 3
%-----------------------------------------------------------------


%-----------------------------------------------------------------
% Awal BAB 4
%-----------------------------------------------------------------
\chapter{ANALISIS DAN PERANCANGAN SISTEM}
\label{ANALISIS DAN PERANCANGAN SISTEM}

	\section{Deskripsi Umum Sistem}
	\label{rancangan deskripsi umum sistem}
	Analisis sistem terdiri dari gambaran umum sistem yang dapat dilihat pada bagian 4.1 dan analisis kebutuhan sistem yang dapat dilihat pada bagian 4.2.


	\section{Analisis Kebutuhan Sistem}
	\label{rancangan analisis kebutuhan sistem}
	Dalam membangun sistem ini, diperlukan analisa kebutuhan fungsional. Kebutuhan fungsional adalah kebutuhan yang berkaitan dengan fungsi-fungsi
yang harus ada dalam sistem. Serta akan dijelaskan kebutuhan perangkat keras dan perangkat lunak yang dibutuhkan dalam membangun sistem ini.

	\section{Pembuatan Sistem}
	\label{rancangan pembuatan sistem}

		\subsection{Pembuatan Sistem Pengenalan Entitas Bernama}
		\label{rancangan pembuatan sistem pengenalan entitas bernama}
		\input{BAB_TESIS/BAB4/3_2_SISTEM_PENGENALAN_ENTITAS_BERNAMA}

		\subsection{Pembuatan Sistem Ekstraksi Kalimat Pernyataan}
		\label{rancangan sistem ekstraksi kalimat pernyataan}
		\input{BAB_TESIS/BAB4/3_3_SISTEM_EKSTRAKSI_KALIMAT_PERNYATAAN}

	\section{Rancangan Antarmuka}
	\label{rancangan antarmuka}

		\subsection{Deskripsi}
		\label{rancangan deskripsi antarmuka}
		\input{BAB_TESIS/BAB4/4_1_DESKRIPSI_RANCANGAN_ANTARMUKA}

		\subsection{\textit{Wireframe}}
	    \label{rancangan wireframe antarmuka}
	    \input{BAB_TESIS/BAB4/4_2_WIREFRAME_ANTARMUKA}

%-----------------------------------------------------------------
% Akhir BAB 4
%-----------------------------------------------------------------


%-----------------------------------------------------------------
% Awal BAB 5
%-----------------------------------------------------------------
\chapter{IMPLEMENTASI SISTEM}
\label{IMPLEMENTASI SISTEM}

	\section{Spesifikasi}
	\label{implementasi spesifikasi}
	\input{BAB_TESIS/BAB5/1_SPESIFIKASI}

	\section{Implementasi Sistem Pengenalan Entitas Bernama}
	\label{implementasi sistem ner}
	\input{BAB_TESIS/BAB5/2_1_IMPLEMENTASI_SISTEM_NER}

	\section{Implementasi Sistem Ekstraksi Kalimat Pernyataan}
	\label{implementasi sistem ekstraksi kalimat pernyataan}
	\input{BAB_TESIS/BAB5/2_2_IMPLEMENTASI_SISTEM_EKTRAKSI_KALIMAT_PERNYATAAN}

%-----------------------------------------------------------------
% Akhir BAB 5
%-----------------------------------------------------------------



%-----------------------------------------------------------------
% Awal BAB 6
%-----------------------------------------------------------------
\chapter{PENGUJIAN DAN PEMBAHASAN SISTEM}
\label{PENGUJIAN DAN PEMBAHASAN SISTEM}
\input{BAB_TESIS/BAB6/1_PENDAHULUAN}

	\section{Pengujian Sistem Pengenalan Entitas Bernama}
	\label{pengujian sistem ner}
	\input{BAB_TESIS/BAB6/2_PENGUJIAN_SISTEM_NER}

	\section{Pengujian Sistem Ekstraksi Kalimat Pernyataan}
	\label{pengujian sistem ekstraksi kalimat pernyataan}
	\input{BAB_TESIS/BAB6/3_PENGUJIAN_SISTEM_EKSTRAKSI_KALIMAT_PERNYATAAN}

%-----------------------------------------------------------------
% Akhir BAB 6
%-----------------------------------------------------------------


%-----------------------------------------------------------------
% Awal BAB 7
%-----------------------------------------------------------------
\chapter{PENUTUP}
\label{PENUTUP}

	\section{Kesimpulan}
	\label{penutup kesimpulan}
	Kesimpulan dari penelitian ini adalah sebagai berikut:
\begin{enumerate}
	\item Model yang dihasilkan belum dapat mengklasifikasi risiko autentikasi dengan baik. Sistem autentikasi M2M berbasis risiko menggunakan Random Forest dapat mengklasikasi risiko autentikasi. Dengan akurasi 70.8\%, presisi 70.1\%, \textit{recall} 96.8\%, dan \textit{F1-score} 71.3\%. Ketimpangan akurasi dan \textit{recall} disebabkan oleh ketidakseimbangan jumlah data pada kelas yang berbeda.
	\item Pembatasan fitur kepentingan dapat berpengaruh pada akurasi sistem.
\end{enumerate}


	\section{Saran}
	\label{penutup saran}
	Penelitian ini masih memiliki beberapa kekurangan yang dapat diperbaiki pada penelitian selanjutnya, yaitu:
\begin{enumerate}
	\item Penelitian ini masih menggunakan dataset hybrid. Sehingga perlu dilakukan penelitian lebih lanjut dengan menggunakan dataset asli.
	\item Akurasi sistem masih dapat ditingkatkan, serta perlu dilakukan penelitian lebih lanjut untuk meningkatkan keamanan sistem.
	\item Opitimasi parameter Random Forest masih dapat dilakukan lebih lanjut.
	\item Dapat dilakukan perbandingan dengan memilih target parameter yang berbeda.
\end{enumerate}


%-----------------------------------------------------------------
% Akhir BAB 7
%-----------------------------------------------------------------

%-----------------------------------------------------------------
% Awal Daftar Pustaka
%-----------------------------------------------------------------
\begin{thebibliography}{99}
    \addcontentsline{toc}{chapter}{DAFTAR PUSTAKA}
    
    \bibitem[Agarwal et al.(2016)]{Agarwal2016}
    Agarwal, L., Khan, H., \& Hengartner, U. (2016). Ask Me Again But Don’t Annoy Me: Evaluating Re-authentication Strategies for Smartphones. 221–236. \url{https://www.usenix.org/conference/soups2016/technical-sessions/presentation/agarwal}
    
    \bibitem[Alam \& Vuong(2013)]{Alam2013}
    Alam, M. S., \& Vuong, S. T. (2013). Random Forest Classification for Detecting Android Malware. 2013 IEEE International Conference on Green Computing and Communications and IEEE Internet of Things and IEEE Cyber, Physical and Social Computing, 663–669. \url{https://doi.org/10.1109/greencom-ithings-cpscom.2013.122}
    
    \bibitem[Braunstein(2022)]{Braunstein2022}
    Braunstein, Mark L. (2022). FHIR. Computers in Health Care, 233–291. \url{https://doi.org/10.1007/978-3-030-91563-6_9}
    
    \bibitem[Cabarcos et al.(2019)]{Cabarcos2019}
    Cabarcos, P. A., Arias-Cabarcos, P., Krupitzer, C., \& Becker, C. (2019). A Survey on Adaptive Authentication. ACM Computing Surveys, 52(4), 80. \url{https://doi.org/10.1145/3336117}
    
    \bibitem[Doerfler et al.(2019)]{Doerfler2019}
    Doerfler, P., Thomas, K., Marincenko, M., Ranieri, J., Jiang, Y., Moscicki, A., \& McCoy, D. (2019). Evaluating Login Challenges as a Defense Against Account Takeover. 372–382. \url{https://doi.org/10.1145/3308558.3313481}
    
    \bibitem[Dutson et al.(2019)]{Dutson2019}
    Dutson, J., Allen, D., Eggett, D. L., \& Seamons, K. E. (2019). Don’t Punish all of us: Measuring User Attitudes about Two-Factor Authentication. 119–128. \url{https://doi.org/10.1109/eurospw.2019.00020}
    
    \bibitem[Feth dkk (2019)]{Feth2019}
    Feth, Denis, dan Svenja Polst. “Heuristics and Models for Evaluating the Usability of Security Measures.” Dalam Proceedings of Mensch und Computer 2019, 275–85. MuC ’19. New York, NY, USA: Association for Computing Machinery, 2019. doi:10.1145/3340764.3340789.
    
    \bibitem[Misbahuddin et al.(2017)]{Misbahuddin2017}
    Misbahuddin, M., B. S. Bindhumadhava, B. S. Bindhumadhava, Bindhumadhava, B. S., \& Dheeptha, B. (2017). Design of a risk-based authentication system using machine learning techniques. 1–6. \url{https://doi.org/10.1109/uic-atc.2017.8397628}
    
    \bibitem[Prasad et al.(2017)]{Prasad2017}
    Prasad, K. K., K, K. P., \& Aithal, S. (2017). A Study on Enhancing Mobile Banking Services Using Location Based Authentication. \url{https://doi.org/10.47992/ijmts.2581.6012.0006}
    
    \bibitem[Rahat et al.(2021)]{Rahat2021}
    Rahat, Tamjid Al, Feng, Yu, \& Tian, Yuan. (2021). Cerberus. Cornell University - ArXiv. \url{https://doi.org/10.1145/3548606.3559381}
    
    \bibitem[Roy \& Dasgupta(2018)]{Roy2018}
    Roy, A., \& Dasgupta, D. (2018). A fuzzy decision support system for multifactor authentication. Soft Computing - A Fusion of Foundations, Methodologies and Applications, 22(12), 3959–3981. \url{https://doi.org/10.1007/s00500-017-2607-6}
    
    \bibitem[Solapurkar(2016)]{Solapurkar2016}
    Solapurkar, P. (2016). Building secure healthcare services using OAuth 2.0 and JSON web token in IOT cloud scenario. International Conferences on Contemporary Computing and Informatics, 99–104. \url{https://doi.org/10.1109/ic3i.2016.7917942}
    
    \bibitem[Speiser et al.(2019)]{Speiser2019}
    Speiser, J. L., Miller, M., Miller, M. E., Tooze, J. A., \& Ip, E. H. (2019). A Comparison of Random Forest Variable Selection Methods for Classification Prediction Modeling. Expert Systems With Applications, 134, 93–101. \url{https://doi.org/10.1016/j.eswa.2019.05.028}
    
    \bibitem[Sujudi, Heryawan et al., 2022]{Sujudi2022}
    Sujudi, Hammam Mahfuzh, dan Lukman Heryawan. “An Automatic Data Mapping for Interoperability of OpenEMR Medical Practice Management Software Using the Fast Healthcare Interoperability Resources.” Advanced Biomedical Engineering 11 (2022): 186–93. doi:10.14326/abe.11.186.
    
    \bibitem[Taneja(2013)]{Taneja2013}
    Taneja, M. (2013). An analytics framework to detect compromised IoT devices using mobility behavior. Information and Communication Technology Convergence, 38–43. \url{https://doi.org/10.1109/ictc.2013.6675302}
    
    \bibitem[Thomas et al.(2017)]{Thomas2017}
    Thomas, K., Li, F., Zand, A., Barrett, J., Ranieri, J., Invernizzi, L., Markov, Y., Comanescu, O., Eranti, V., Moscicki, A., Margolis, D., Paxson, V., \& Bursztein, E. (2017). Data Breaches, Phishing, or Malware?: Understanding the Risks of Stolen Credentials. 1421–1434. \url{https://doi.org/10.1145/3133956.3134067}
    
    \bibitem[Wiefling et al.(2021)]{Wiefling2021}
    Wiefling, Stephan, Markus Dürmuth, \& Luigi Lo Iacono. (2021). What’s in Score for Website Users: A Data-driven Long-term Study on Risk-based Authentication Characteristics. Financial Cryptography. \url{https://doi.org/10.1007/978-3-662-64331-0_19}
    
    \bibitem[Wiefling et al.(2022)]{Wiefling2022}
    Wiefling, Stephan, Paul René Jørgensen, Sigurd Thunem, \& Luigi Lo Iacono. (2022). Pump Up Password Security! Evaluating and Enhancing Risk-Based Authentication on a Real-World Large-Scale Online Service. ACM Transactions on Privacy and Security. \url{https://doi.org/10.1145/3546069}
    
    \bibitem[Zhang et al.(2012)]{Zhang2012}
    Zhang, F., Kondoro, A., \& Muftic, S. (2012). Location-Based Authentication and Authorization Using Smart Phones. 2012 IEEE 11th International Conference on Trust, Security and Privacy in Computing and Communications, 1285–1292. \url{https://doi.org/10.1109/trustcom.2012.198}
\end{thebibliography}

%-----------------------------------------------------------------
% Akhir Daftar Pustaka
%-----------------------------------------------------------------


%-----------------------------------------------------------------
% Awal lampiran
%-----------------------------------------------------------------
\appendix

\chapter{BERKAS JSON UNTUK MODEL SISTEM PENGENALAN ENTITAS BERNAMA}
\label{BERKAS JSON UNTUK MODEL SISTEM PENGENALAN ENTITAS BERNAMA}
\input{BAB_TESIS/BAB9_LAMPIRAN/1_BERKAS_MODEL_NER}

%-----------------------------------------------------------------
% Akhir lampiran
%-----------------------------------------------------------------

\end{document}