Setelah dilakukan encoding, maka seluruh kolom memiliki tipe data numerik. Berikut adalah contoh kode untuk melakukan pemilihan fitur menggunakan Gini Importance.

\begin{lstlisting}
    ### Gini importance 
    # create the classifier with n_estimators = default
    clf = RandomForestClassifier(random_state=0)

    # fit the model to the training set
    clf.fit(X_train, y_train)

    # view the feature scores
    feature_scores = pd.Series(clf.feature_importances_, index=X_train.columns).sort_values(ascending=False)
    
    # Top 10 important features
    feature_scores.head(10) 
    \end{lstlisting}

    Pada kode di atas dilakukan pemilihan 10 fitur teratas. Dikarenakan jumlah fitur yang banyak, setelah dilakukkan encoding maka akan sulit untuk memvisualisasikan seluruh fitur.
    Berikut adalah hasil keluaran dari tahap ini.

    \begin{table}[H]
    \caption{Gini Importance of Each Feature}

    \centering
    \begin{tabular}{|l|l|}
    \hline
    \textbf{Feature} & \textbf{Gini Importance} \\ \hline
    asn & 0.017551 \\ 
    country\_2 & 0.009943 \\ 
    country\_4 & 0.004708 \\ 
    country\_6 & 0.003670 \\ 
    ip\_address\_23 & 0.003618 \\ 
    os\_detail\_1 & 0.003317 \\ 
    browser\_1 & 0.002975 \\ 
    os\_detail\_16 & 0.002832 \\ 
    user\_agent\_string\_49 & 0.002508 \\ 
    browser\_2 & 0.002213 \\ \hline
    \end{tabular}
    \label{tab:gini_importance}
    \end{table}

    Dalam tabel \ref{tab:gini_importance}, jika di lakukkan pengelompokkan maka akan terlihat bahwa fitur 'asn', 'country', 'ip\_address', 'os\_detail', 'browser', dan 'user\_agent\_string' memiliki nilai Gini Importance yang tinggi. 
    Namun, hanya 4 group teratas yang memiliki nilai Gini Importance yang tinggi, yaitu 'asn', 'country', 'ip\_address', dan 'os\_detail' yang akan digunakan sebagai fitur dalam pembuatan model.