Pada bagian ini akan dijelaskan mengenai implementasi pemilihan fitur. Pemilih-
an fitur dilakukan dengan cara memilih fitur yang memiliki korelasi tinggi dengan target.
Berikut adalah tahapan pemilihan fitur.

Sebelum itu dilakukan eksplorasi data untuk mengetahui jumlah baris dan kolom, tipe data, dan statistik deskriptif dari data. 
Tahap ini diperlukan untuk mengetahui tipe data dari setiap kolom. 

Asumsi yang digunakan adalah kolom yang memiliki tipe data numerik memiliki korelasi yang lebih tinggi dibandingkan dengan kolom yang memiliki tipe data string.
Berikut adalah contoh kode untuk mengetahui tipe data dari setiap kolom.

\begin{lstlisting}
    categorical = [var for var in df.columns if df[var].dtype=='O']
    print('There are {} categorical variables\n'.format(len(categorical)))
    print('The categorical variables are :\n\n', categorical)

    There are 8 categorical variables

    The categorical variables are :
    ['login_timestamp', 'ip_address', 'country', 'region', 'city', 'user_agent_string', 'browser', 'os_detail']
    \end{lstlisting}

    Berikut adalah hasil keluaran dari tahap ini.

    \begin{table}[H]
        \caption{Data Type of Each Column}
        \centering
        \begin{tabular}{|l|l|}
        \hline
        \textbf{Column Name} & \textbf{Data Type} \\ \hline
        login\_timestamp & object \\ 
        ip\_address & object \\ 
        country & object \\ 
        region & object \\ 
        city & object \\ 
        asn & int64 \\ 
        user\_agent\_string & object \\ 
        browser & object \\ 
        os\_detail & object \\ 
        is\_login\_success & bool \\ \hline
        \end{tabular}
        \label{tab:data_type}
        \end{table}

        Berdasarkan tabel \ref{tab:data_type}, terlihat bahwa kolom 'ASN' memiliki tipe data numerik, sedangkan kolom lainnya memiliki tipe data string.

