Hal ini dilakukan untuk membatasi jumlah dataset dan device type yang bertujuan mengurangi waktu komputasi dalam pembuatan model. Berikut adalah contoh kode untuk melakukan penyaringan user agent dan device type. 

\begin{lstlisting}
    # check lenght in column user_agent_string
    features['length'] = features['user_agent_string'].apply(
        lambda row: min(len(row), len(row)) if isinstance(row, str) else None
    )
    print(features['length'].mean())
    \end{lstlisting}

    Kode di atas digunakan untuk mengetahui panjang rata-rata string pada kolom 'User Agent String'. Hasilnya adalah 136.652141700553. 
    Setelah itu dilakukan penyaringan data dengan cara menghapus data yang memiliki panjang string lebih dari 136. Berikut adalah contoh kode untuk melakukan penyaringan data.

\begin{lstlisting}
    # only keep rows with device type desktop
    features = features[features.device_type == 'desktop']
    # filter the DataFrame based on the length of column 'user_agent_string'
    features = features[features['user_agent_string'].str.len() < 136]
    \end{lstlisting}

    Setelah itu dilakukan penyaringan data dengan cara menghapus data yang memiliki device type selain 'desktop'.
