Studi ini menggunakan pendekatan berbasis risiko untuk mengidentifikasi dan menilai potensi risiko terkait otentikasi Machine-to-Machine (M2M). Analisis pelaku ancaman, kerentanan, dan dampak serangan dilakukan, serta evaluasi metode otentikasi M2M saat ini. Penelitian ini mengembangkan strategi peningkatan otentikasi guna mengurangi risiko serangan. Peningkatan perangkat IoT dalam teknologi perawatan kesehatan menekankan pentingnya otentikasi yang aman antar perangkat. Berbagai pendekatan seperti model enkripsi dan protokol otentikasi telah diusulkan, namun kurang dalam pendekatan berbasis risiko yang mengintegrasikan analisis ancaman dengan evaluasi efektivitas metode otentikasi.

Penelitian ini memanfaatkan algoritma Random Forest untuk mengklasifikasikan akses dan menilai efektivitas metode otentikasi saat ini. Temuan penting termasuk identifikasi risiko akses perangkat tidak sah seperti \textit{replay attack}. Dalam konteks Fast Healthcare Interoperability Resources (FHIR), studi ini mencapai akurasi 0,708, presisi 0,701, recall 0,968, dan skor F1 0,813 dengan algoritma Random Forest.

Sebagai perbandingan, studi menggunakan Local Outlier Factor (LOF) untuk autentikasi berbasis data swipe pengguna smartphone menunjukkan bahwa meskipun model belum dioptimalkan mampu mencapai tingkat keberhasilan lebih dari 90\% bahkan dengan FAR hingga 40\%. Ini menunjukkan bahwa LOF dapat memberikan metrik autentikasi kompetitif dengan model kompleks seperti Random Forest. 

Penelitian ini memberikan wawasan berharga bagi organisasi yang menerapkan perangkat IoT, khususnya di sektor teknologi perawatan kesehatan, untuk mengurangi risiko terkait otentikasi M2M.


Kata kunci : Otentikasi Machine-to-Machine (M2M), Fast Healthcare Interoperability Resources (FHIR), Random Forest, Local Outlier Factor (LOF), Perbaikan Autentikasi