Studi ini menggunakan pendekatan berbasis risiko untuk mengidentifikasi dan menilai potensi risiko yang terkait dengan otentikasi M2M. Ini melibatkan identifikasi pelaku ancaman potensial, kerentanan, dan dampak dari serangan yang berhasil. Studi ini juga mengevaluasi metode otentikasi M2M saat ini dan keefektifannya dalam mengurangi risiko yang teridentifikasi. Terakhir, penelitian ini merekomendasikan strategi untuk meningkatkan otentikasi M2M untuk mengurangi risiko serangan yang berhasil. Studi ini diharapkan dapat mengidentifikasi beberapa risiko yang terkait dengan otentikasi M2M, antara lain:

Akses tidak sah ke perangkat: Peretas dapat mengeksploitasi kerentanan dalam autentikasi M2M untuk mendapatkan akses tidak sah ke perangkat dan mencuri informasi sensitif.
Serangan penolakan layanan: Penyerang dapat meluncurkan serangan penolakan layanan untuk mengganggu komunikasi M2M dan menyebabkan downtime sistem.

Studi ini memberikan wawasan berharga tentang risiko yang terkait dengan otentikasi M2M dan strategi untuk memitigasi risiko tersebut. Temuan penelitian ini berguna untuk organisasi yang menerapkan perangkat IoT khususnya pada sektor teknologi kesehatan.