Studi ini menggunakan pendekatan berbasis risiko untuk mengidentifikasi dan menilai potensi risiko yang terkait dengan otentikasi Machine-to-Machine (M2M). Hal ini melibatkan analisis pelaku ancaman potensial, kerentanan, dan dampak serangan yang berhasil. Studi ini juga mengevaluasi metode otentikasi M2M saat ini dan efektivitasnya dalam memitigasi risiko yang teridentifikasi, menggunakan algoritma Random Forest untuk mengklasifikasikan akses. Selain itu, penelitian ini mengusulkan strategi untuk meningkatkan otentikasi M2M guna mengurangi risiko serangan yang berhasil.

Temuan ini diharapkan dapat menyoroti beberapa risiko yang terkait dengan otentikasi M2M, seperti akses perangkat yang tidak sah, di mana peretas mengeksploitasi kerentanan untuk mencuri informasi sensitif, dan serangan penolakan layanan (DoS), yang mengganggu komunikasi M2M dan menyebabkan waktu henti sistem. Dalam konteks Fast Healthcare Interoperability Resources (FHIR), hal ini penting karena FHIR digunakan untuk bertukar data kesehatan elektronik antara sistem dan perangkat medis. Penelitian ini mencapai akurasi 0,708, presisi 0,701, recall 0,968, dan skor F1 0,813 dengan pengklasifikasi Random Forest.

Studi ini memberikan wawasan berharga tentang risiko yang terkait dengan otentikasi M2M dan menawarkan strategi mitigasi. Temuan penelitian ini sangat berguna bagi organisasi yang menerapkan perangkat IoT, khususnya di sektor teknologi perawatan kesehatan. Dengan mengatasi kerentanan dan mengusulkan metode autentikasi yang efektif, penelitian ini bertujuan untuk secara signifikan mengurangi risiko terkait autentikasi M2M di lingkungan sensitif seperti layanan kesehatan.