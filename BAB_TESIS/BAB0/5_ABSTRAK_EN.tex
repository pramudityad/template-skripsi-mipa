This study employs a risk-based approach to identify and assess potential risks associated with Machine-to-Machine (M2M) authentication. Threat actor analysis, vulnerability assessment, and evaluation of the impact of attacks are conducted, along with an evaluation of current M2M authentication methods. The research develops strategies to enhance authentication to mitigate attack risks. The increase in IoT devices in healthcare technology emphasizes the importance of secure inter-device authentication. Various approaches such as encryption models and authentication protocols have been proposed, but they fall short of a risk-based approach that integrates threat analysis with the evaluation of authentication effectiveness.

This study utilizes the Random Forest algorithm to classify access and assess the effectiveness of current authentication methods. Key findings include the identification of risks such as unauthorized device access like replay attacks. In the context of Fast Healthcare Interoperability Resources (FHIR), this study achieves an accuracy of 0.708, a precision of 0.701, a recall of 0.968, and an F1 score of 0.813 using the Random Forest algorithm.

For comparison, a study using the Local Outlier Factor (LOF) for swipe-based user authentication on smartphones showed that even without optimization, the model could achieve success rates of over 90\% with a FAR of up to 40\%. This indicates that LOF can provide competitive authentication metrics with complex models like Random Forest.

This research offers valuable insights for organizations implementing IoT devices, particularly in the healthcare technology sector, to reduce risks associated with M2M authentication.

Keywords: Machine-to-Machine (M2M) Authentication, Fast Healthcare Interoperability Resources (FHIR), Random Forest, Local Outlier Factor (LOF), Authentication Improvement