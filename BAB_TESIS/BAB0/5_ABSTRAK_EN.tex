This study employs a risk-based approach to identify and assess potential risks associated with M2M authentication. It involves identifying potential threat actors, vulnerabilities, and the impacts of successful attacks. The study also evaluates current M2M authentication methods and their effectiveness in reducing identified risks. Lastly, this research recommends strategies to enhance M2M authentication to mitigate successful attack risks. The study is expected to identify several risks associated with M2M authentication, including:

Unauthorized access to devices: Hackers can exploit vulnerabilities in M2M authentication to gain unauthorized access to devices and steal sensitive information.
Denial of service attacks: Attackers can launch denial of service attacks to disrupt M2M communication and cause system downtime.

This study provides valuable insights into the risks associated with M2M authentication and strategies to mitigate these risks. The research findings are useful for organizations implementing IoT devices, particularly in the healthcare technology sector.