This study employs a risk-based approach to identify and assess potential risks associated with Machine-to-Machine (M2M) authentication. It involves analyzing potential threat actors, vulnerabilities, and the impacts of successful attacks. The study also evaluates current M2M authentication methods and their effectiveness in mitigating identified risks, using a Random Forest algorithm to classify access. Additionally, the research proposes strategies to enhance M2M authentication to reduce the risk of successful attacks.

The findings are expected to highlight several risks associated with M2M authentication, such as unauthorized device access, where hackers exploit vulnerabilities to steal sensitive information, and denial of service (DoS) attacks, which disrupt M2M communications and cause system downtime. In the context of Fast Healthcare Interoperability Resources (FHIR), this is crucial as FHIR is used to exchange electronic health data between medical systems and devices. The study achieved an accuracy of 0.708, a precision of 0.701, a recall of 0.968, and an F1 score of 0.813 with the Random Forest classifier.

This study provides valuable insights into the risks associated with M2M authentication and offers strategies for mitigation. The research findings are especially useful for organizations implementing IoT devices, particularly in the healthcare technology sector. By addressing vulnerabilities and proposing effective authentication methods, the study aims to significantly reduce the risks linked to M2M authentication in sensitive environments such as healthcare.