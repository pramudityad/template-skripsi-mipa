FHIR, singkatan dari Fast Healthcare Interoperability Resources, merupakan standar internasional yang diperkenalkan oleh Health Level Seven International (HL7) untuk memfasilitasi pertukaran data kesehatan elektronik. Standar ini dirancang untuk mengatasi tantangan interoperabilitas antara sistem-sistem informasi kesehatan yang beragam, dengan tujuan memungkinkan pertukaran data yang cepat, fleksibel, dan terstandarisasi di seluruh industri kesehatan. 

FHIR menggunakan format data yang ringan seperti JSON atau XML, dan protokol komunikasi web standar seperti HTTP atau HTTPS, yang memfasilitasi integrasi dengan sistem-sistem modern dengan lebih mudah. Dengan pendekatan moduler, FHIR memungkinkan akses granular terhadap informasi kesehatan, sesuai kebutuhan aplikasi atau pengguna. 

Adopsi FHIR diharapkan dapat meningkatkan interoperabilitas di seluruh rantai perawatan kesehatan, memungkinkan pertukaran informasi yang lebih efisien dan akurat, serta mendukung pengembangan aplikasi kesehatan yang inovatif dan terintegrasi. Sebagai hasilnya, FHIR juga membuka pintu bagi pengembangan solusi-solusi teknologi kesehatan yang lebih canggih, seperti analisis big data dan kecerdasan buatan, serta integrasi dengan perangkat medis wearable. 