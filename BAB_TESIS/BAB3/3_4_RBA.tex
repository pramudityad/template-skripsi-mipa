Risk-based adalah suatu metode yang digunakan untuk mengukur dan
mengelola risiko. Dalam konteks keamanan, risk-based authentication adalah metode autentikasi yang mengukur tingkat risiko dari suatu permintaan akses, dan mengambil tindakan yang sesuai berdasarkan tingkat risiko tersebut. Metode ini bertujuan untuk mengenali dan menangani ancaman potensial tanpa mengekang fleksibilitas dan kenyamanan pengguna.
Dalam konteks Machine-to-Machine (M2M) authentication, risk-based authentication digunakan untuk mengukur tingkat risiko dari suatu permintaan akses dan mengambil tindakan yang sesuai berdasarkan tingkat risiko tersebut.
Prosesnya dapat dilakukan dengan cara menganalisis faktor-faktor yang dapat meningkatkan risiko, seperti lokasi geografis, waktu akses, dan jenis perangkat yang digunakan.
Setelah tingkat risiko diukur, sistem dapat mengambil tindakan yang sesuai. Jika tingkat risiko dianggap rendah, maka autentikasi dapat dilakukan secara otomatis tanpa intervensi manusia. Namun, jika tingkat risiko dianggap tinggi, maka autentikasi dapat dilakukan dengan cara yang lebih ketat, seperti mengharuskan verifikasi melalui kode SMS atau panggilan telepon, atau pembatasan akses sesuai dengan level risiko.
Risk-based authentication juga dapat digabungkan dengan metode analisis risiko dinamis, yaitu mengukur risiko secara real-time dan mengambil tindakan sesuai dengan situasi yang ada. Ini dapat membantu sistem untuk mengenali dan menangani ancaman potensial secara efektif tanpa mengekang fleksibilitas dan kenyamanan pengguna seperti ilustrasi pada Gambar 3.2.

Bagian ini membahas pertimbangan etis penelitian dan potensi masalah serta
keterbatasannya. Jika menyangkut penelitian dengan makhluk hidup, maka dibutuhkan adanya \textit{ethical clearance}, di bagian ini hal itu akan dibahas. Demikian juga tentang keterbatasan ataupun masalah yang akan timbul.