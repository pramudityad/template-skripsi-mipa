Hubungan antara autentikasi dan FHIR berkaitan dengan keamanan dan akses kontrol dalam pertukaran data kesehatan elektronik. Autentikasi digunakan untuk memverifikasi identitas entitas yang terlibat dalam pertukaran data menggunakan standar FHIR. Setelah identitas tersebut diverifikasi, otorisasi diterapkan untuk menentukan hak akses entitas tersebut terhadap data yang disediakan oleh layanan FHIR.

Dalam konteks FHIR, autentikasi digunakan untuk memastikan bahwa entitas yang mencoba mengakses atau menyediakan data kesehatan melalui API FHIR adalah entitas yang sah. Ini bisa berarti memverifikasi identitas pengguna, aplikasi, atau sistem yang berusaha berinteraksi dengan layanan FHIR. Autentikasi bisa dilakukan menggunakan berbagai metode, seperti kata sandi, token, atau mekanisme autentikasi yang lebih kuat seperti sertifikat digital atau biometrik, tergantung pada kebutuhan dan kebijakan keamanan sistem.

Setelah autentikasi berhasil dilakukan, otorisasi diterapkan untuk menentukan apa yang diizinkan entitas tersebut lakukan dengan data yang tersedia melalui layanan FHIR. Misalnya, seorang dokter mungkin memiliki akses penuh untuk melihat dan mengubah catatan medis pasien tertentu, sementara seorang petugas administrasi hanya diizinkan untuk melihat informasi dasar pasien tanpa memiliki kemampuan untuk mengubahnya. Otorisasi dalam konteks FHIR memastikan bahwa akses ke data kesehatan dikontrol sesuai dengan kebutuhan dan kebijakan privasi yang berlaku.

Dengan demikian, autentikasi dan otorisasi berperan penting dalam menjaga keamanan dan kerahasiaan data kesehatan yang ditangani oleh layanan FHIR, memastikan bahwa hanya entitas yang berwenang yang dapat mengakses informasi yang sensitif dan penting tersebut.