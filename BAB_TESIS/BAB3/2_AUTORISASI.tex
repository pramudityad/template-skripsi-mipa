Otorisasi merujuk pada proses yang menentukan hak akses yang diberikan kepada entitas setelah autentikasi identitasnya berhasil dilakukan. Otorisasi memainkan peran penting dalam mengatur akses ke sumber daya dan layanan di dalam suatu sistem. Ini melibatkan penentuan apakah subjek atau entitas memiliki izin yang sesuai untuk melakukan tindakan tertentu dalam lingkungan yang diberikan. Proses otorisasi sering kali dilakukan setelah proses autentikasi yang sukses, di mana autentikasi memverifikasi identitas entitas. Dengan adanya otorisasi, sistem dapat memastikan bahwa hanya entitas yang memiliki hak yang sesuai yang diberikan akses ke sumber daya atau layanan tertentu, yang pada gilirannya membantu menjaga keamanan sistem secara keseluruhan. Misalnya, dalam sebuah aplikasi perbankan, setelah seorang pengguna berhasil mengautentikasi identitasnya, proses otorisasi akan menentukan hak akses pengguna tersebut terhadap fungsi-fungsi seperti pengecekan saldo, transfer dana, atau pembayaran tagihan. Oleh karena itu, pemahaman yang mendalam tentang konsep otorisasi penting untuk merancang dan mengimplementasikan sistem informasi yang aman dan efektif.
