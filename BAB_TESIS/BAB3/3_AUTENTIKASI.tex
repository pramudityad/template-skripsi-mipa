Autentikasi adalah konsep fundamental yang diperlukan untuk memvalidasi keaslian identitas entitas tertentu dalam suatu sistem. Identitas, sebagai inti dari autentikasi, merujuk pada informasi yang digunakan untuk mengidentifikasi subjek. Kredensial, sebagai elemen kunci dalam proses autentikasi, terdiri dari informasi otentikasi yang diperlukan untuk membuktikan identitas subjek, seperti kata sandi, token, atau biometrik.

Metode autentikasi beragam dan dapat mencakup kata sandi, token, biometrik, sertifikat digital, serta otorisasi multi-faktor (MFA). Protokol autentikasi, sebagai serangkaian langkah atau aturan, memberikan panduan bagi pelaksanaan autentikasi dalam suatu sistem, contohnya OAuth, OpenID, SAML, dan Kerberos.

Keamanan merupakan aspek krusial dalam autentikasi, yang mencakup kerahasiaan kredensial, integritas data autentikasi, dan non-repudiasi. Pemahaman akan kelemahan dan ancaman terhadap sistem autentikasi, seperti serangan phishing, brute force, dan man-in-the-middle, penting untuk meningkatkan ketahanan sistem.

Selain itu, autentikasi harus dapat diandalkan, sehingga sistem dapat memberikan verifikasi identitas yang konsisten dan akurat