Hasil pengujian waktu data training berisi hasil pengujian terhadap waktu yang dibutuhkan oleh sistem untuk melakukan proses data training. Pengujian ini dilakukan dengan cara membandingkan waktu yang dibutuhkan oleh sistem untuk melakukan proses data training.
Dalam pengujian ini, dilakukkan pengujian untuk beberapa ukuran dataset. Ukuran dataset yang digunakan adalah 10000, 20000, 30000, 40000 dan 50000. 
Berikut adalah hasil pengujian waktu data training dapat dilihat pada \ref*{table:waktu_rf}.

\begin{table}[H]
    \caption{Hasil Pengujian Waktu Data Training Random Forest}
    \centering
    \begin{tabular}{|c|c|}
    \hline
    \textbf{Ukuran Dataset} & \textbf{Waktu Data Training (detik)} \\
    \hline
    10000 & 453 \\
    \hline
    20000 & 901 \\
    \hline
    30000 & 1937 \\
    \hline
    40000 & 2237 \\
    \hline
    50000 & error \\
    \hline
    \end{tabular}
    \label{table:waktu_rf}
\end{table}