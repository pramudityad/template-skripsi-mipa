\textbf{Analisis Kuantitatif:} Heuristic Authentication memiliki beberapa keunggulan dalam konteks autentikasi mesin ke mesin. Pertama, pendekatan ini sangat sederhana dan mudah diimplementasikan karena mengandalkan aturan-aturan sederhana yang telah ditentukan sebelumnya. Selain itu, Heuristic Authentication tidak memerlukan sumber daya komputasi yang besar, sehingga cocok untuk lingkungan dengan keterbatasan sumber daya.

\textbf{Analisis Kualitatif:} Namun, ada beberapa batasan yang perlu dipertimbangkan dalam menggunakan Heuristic Authentication. Pertama, pendekatan ini mungkin kurang cocok untuk kasus-kasus autentikasi yang kompleks atau data yang tidak terstruktur dengan baik. Selain itu, aturan-aturan heuristik mungkin tidak cukup fleksibel untuk menangani variasi perilaku pengguna yang kompleks.

Feth dkk (2019) Secara kualitatif, para pengembang merespons positif terhadap pendekatan yang digunakan dalam penelitian ini. Mereka menyatakan bahwa mereka menemukan masalah baru atau dorongan pemikiran yang dapat membantu dalam pengembangan lebih lanjut. Namun, terdapat perbedaan pendapat mengenai tingkat detail dari model yang digunakan. Beberapa partisipan merasa bahwa model terlalu rinci untuk sebagian besar heuristik.

\textbf{Kelebihan:} Salah satu kelebihan utama Heuristic Authentication adalah kesederhanaan implementasinya. Pendekatan ini mudah diimplementasikan tanpa memerlukan analisis yang rumit atau data pelatihan yang besar. Selain itu, Heuristic Authentication tidak memerlukan sumber daya komputasi yang besar. Serta mayoritas partisipan pada penelitian yang dilakukkan Feth dkk (2019) menyatakan bahwa mereka menemukan model yang digunakan membantu dalam evaluasi usability langkah-langkah keamanan dalam perangkat lunak mereka

\textbf{Kelemahan:} Namun, kekurangan utama Heuristic Authentication adalah kurangnya fleksibilitas. Pendekatan ini tidak dapat menangani kasus-kasus autentikasi yang kompleks dengan baik dan mungkin kurang akurat dalam mengidentifikasi aktivitas yang mencurigakan atau autentikasi yang tidak sah dibandingkan dengan metode yang lebih canggih.

Selain itu, Feth dkk (2019) mengemukakan terdapat perbedaan pendapat di antara partisipan mengenai tingkat detail dari model yang digunakan, menunjukkan bahwa model tersebut mungkin perlu disesuaikan lebih lanjut untuk mencapai kesesuaian yang ideal.

Dengan mempertimbangkan kedua metode ini, organisasi harus memilih berdasarkan pada kebutuhan spesifik, sumber daya yang tersedia, dan tingkat keamanan yang diinginkan. Dalam beberapa kasus, kombinasi dari kedua metode ini juga dapat digunakan untuk meningkatkan keamanan autentikasi.
