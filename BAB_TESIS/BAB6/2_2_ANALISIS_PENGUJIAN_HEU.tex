\textbf{Analisis Kuantitatif:} Penelitian Kalantzis (2022) mencakup sejumlah besar eksperimen yang terperinci, termasuk eksperimen untuk pengguna dan penyerang. Sebagai contoh, satu bagian mencakup 30 eksperimen untuk pengguna dan 1190 untuk penyerang.

\textbf{Analisis Kualitatif:} dengan penggunaan metrik yang terperinci seperti False Acceptance Rate (FAR) dan False Rejection Rate (FRR), yang dilaporkan pada berbagai konfigurasi dan pengguna. Paper ini juga membandingkan kinerja metodologi yang diusulkan dengan model lain melalui analisis komparatif dan menggunakan grafis untuk menggambarkan peningkatan dalam metrik FAR dan FRR secara visual. Seperti yang digunakan pada peneliatn Feth (2019)

\textbf{Kelebihan:} Keuntungan (pros) dari pendekatan ini termasuk pengujian yang sangat komprehensif, yang memberikan keandalan tinggi pada temuan yang dihasilkan, serta metodologi yang menunjukkan peningkatan yang jelas dalam metrik FAR sebesar 29\% dan FRR sebesar 11\%. Metode yang rinci dan penggunaan tabel serta grafik untuk menyajikan data eksperimen membantu pemahaman tentang peningkatan yang dicapai

\textbf{Kelemahan:} Namun, ada beberapa kelemahan (cons) yang perlu diperhatikan. Salah satunya adalah bias model autoencoder terhadap prediksi yang positif, yang dapat mempengaruhi kemampuan generalisasi model. Selain itu, meskipun ada peningkatan signifikan dalam FAR, peningkatan FRR kurang mengejutkan, yang menunjukkan variabilitas dalam kinerja model. Nilai awal beberapa metrik sudah rendah, membatasi potensi peningkatan lebih lanjut, terutama pada FRR.
