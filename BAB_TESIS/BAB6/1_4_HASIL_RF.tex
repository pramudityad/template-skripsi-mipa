Hasil pengujian random forest berisi hasil pengujian terhadap random forest. Pada pengujian pertama ingin dilihat bagaimana performa random forest dengan menggunakan beberapa parameter yang berbeda. Pada pengujian kedua ingin dilihat bagaimana performa random forest dengan menggunakan parameter yang telah dioptimasi.
Dalam percobaan ini dipilih 4 parameter yang akan dioptimasi, yaitu: max\_depth, min\_samples\_leaf, min\_samples\_split, dan n\_estimators. Untuk setiap parameter, akan dicoba beberapa nilai yang berbeda. Untuk setiap kombinasi parameter, akan dilakukan 5 kali percobaan. Untuk setiap percobaan, akan dilakukan 5 kali validasi silang. Dengan demikian, total percobaan yang dilakukan adalah 5 x 5 x 5 x 5 = 625 percobaan.

\begin{table}[h]
    \caption{Parameter Grid}
    \centering
    \begin{tabular}{|c|c|}
    \hline
    \textbf{Parameter} & \textbf{Values} \\
    \hline
    n\_estimators & 100, 200, 500 \\
    \hline
    max\_depth & None, 10, 20 \\
    \hline
    min\_samples\_split & 2, 5, 10 \\
    \hline
    min\_samples\_leaf & 1, 2, 4 \\
    \hline
    \end{tabular}
    \label{table:param_grid}
    \end{table}

Pada Tabel \ref{table:param_grid} ditunjukkan hasil pengujian random forest dengan menggunakan beberapa parameter yang berbeda. Pada tabel \ref{table:sampel_hasil_pengujian_random_forest} ditunjukkan sampel hasil pengujian random forest dengan menggunakan parameter yang telah dioptimasi.
