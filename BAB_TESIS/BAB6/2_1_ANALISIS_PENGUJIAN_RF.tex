\textbf{Analisis Kuantitatif:} Random Forest memiliki beberapa keunggulan dalam konteks autentikasi mesin ke mesin. Pertama, Random Forest dapat memberikan akurasi yang rendah dalam memprediksi perilaku pengguna dan mengidentifikasi aktivitas yang mencurigakan. Algoritme ini efisien dalam pengolahan data kompleks dan tidak linear, cocok untuk situasi di mana terdapat banyak variabel dan interaksi antara variabel tersebut.

\textbf{Analisis Kualitatif:} Di sisi lain, Random Forest mampu menangani pengambilan keputusan yang kompleks dengan memodelkan hubungan yang kompleks antara variabel-variabel dalam data, sehingga cocok untuk kasus-kasus autentikasi yang kompleks. Namun, untuk membangun model yang akurat, Random Forest membutuhkan data yang memadai untuk melatihnya, yang mungkin sulit diperoleh terutama dalam konteks keamanan informasi yang sensitif.

\textbf{Kelebihan:} Salah satu kelebihan utama Random Forest adalah kemampuannya dalam menangani data yang kompleks dan mengurangi risiko overfitting. Cocok untuk situasi di mana terdapat banyak variabel dan interaksi antara variabel tersebut.

\textbf{Kelemahan:} Namun, penggunaan Random Forest juga memiliki beberapa kelemahan. Pertama, memproses ensambel pohon keputusan dapat membutuhkan sumber daya komputasi yang besar. Selain itu, kinerja Random Forest dapat bervariasi tergantung pada pengaturan hyperparameter yang dipilih.
