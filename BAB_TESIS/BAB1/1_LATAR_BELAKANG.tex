Risk-based M2M (Machine-to-Machine) authentication merupakan metode otentikasi yang mengukur tingkat risiko yang terkait dengan suatu perangkat atau sistem dan menyesuaikan tingkat otentikasi yang diperlukan sesuai dengan tingkat risiko tersebut. Dalam sistem kesehatan, FHIR (Fast Healthcare Interoperability Resources) menjadi standar yang digunakan untuk pertukaran informasi kesehatan secara elektronik. Kerangka kerja OAuth menyediakan autentikasi dan otorisasi menggunakan profil dan kredensial pengguna di penyedia identitas yang ada. Hal ini membuat memungkinkan penyerang untuk mengeksploitasi kerentanan apa pun yang timbul dari pertukaran data dengan penyedia. Kerentanan dalam OAuth Alur otorisasi OAuth memungkinkan penyerang untuk mengubah urutan alur normal protokol OAuth (Rahat, Tamjid Al et al., 2021). Sehingga, sistem otentikasi FHIR saat ini hanya didasarkan pada OAuth2 dan OpenID Connect, sehingga risiko dari perangkat yang terhubung tidak diperhitungkan dalam otentikasi.

FHIR adalah sebuah acuan / standar yang digunakan dalam pertukaran informasi tentang kesehatan secara elektronik atau online. FHIR dikembangkan dan diawasi oleh sebuah organisasi yang bernama HL7 (Health Level Seven International) (Mark L. Braunstein, 2022) . HL7 adalah sebuah non-profit organisasi yang menyediakan sebuah framework dan acuan-acuan dalam pertukaran, integrasi, pembagian dan penerimaan informasi tentang kesehatan yang dapat membantu praktik dalam kesehatan, manajemen serta evaluasi pelayanan kesehatan. Dalam konteks FHIR, ini penting karena FHIR digunakan untuk pertukaran data kesehatan elektronik antar sistem dan perangkat medis (Solapurkar, 2016). Karena FHIR digunakan untuk mengakses data kesehatan yang sensitif, penting untuk memastikan bahwa hanya perangkat dan sistem yang sah yang diizinkan untuk mengakses data. Namun, tidak semua perangkat atau sistem memiliki tingkat risiko yang sama (Dutson et al., 2019). Misalnya, perangkat medis yang digunakan untuk mengadministrasikan obat kepada pasien memiliki risiko yang lebih tinggi dibandingkan dengan sensor suhu di ruangan.

Salah satu serangan yang umum terjadi pada kasus autentikasi token ini adalah replay attack , bentuk serangan jaringan di mana transmisi data yang valid diulang atau ditunda secara jahat atau curang. Dengan mengimplementasikan metode autentikasi berbasis risiko (Stephan Wiefling et al., 2021), sistem dapat menyesuaikan tingkat keamanan yang dibutuhkan sesuai dengan tingkat risiko dari perangkat atau sistem yang berkomunikasi, sehingga dapat meningkatkan keamanan dalam pertukaran data kesehatan melalui FHIR.