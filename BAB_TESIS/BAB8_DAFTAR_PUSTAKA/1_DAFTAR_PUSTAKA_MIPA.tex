\begin{thebibliography}{99}
    \addcontentsline{toc}{chapter}{DAFTAR PUSTAKA}
    
    \bibitem[Agarwal et al.(2016)]{Agarwal2016}
    Agarwal, L., Khan, H., \& Hengartner, U. (2016). Ask Me Again But Don’t Annoy Me: Evaluating Re-authentication Strategies for Smartphones. 221–236. \url{https://www.usenix.org/conference/soups2016/technical-sessions/presentation/agarwal}
    
    \bibitem[Alam \& Vuong(2013)]{Alam2013}
    Alam, M. S., \& Vuong, S. T. (2013). Random Forest Classification for Detecting Android Malware. 2013 IEEE International Conference on Green Computing and Communications and IEEE Internet of Things and IEEE Cyber, Physical and Social Computing, 663–669. \url{https://doi.org/10.1109/greencom-ithings-cpscom.2013.122}
    
    \bibitem[Braunstein(2022)]{Braunstein2022}
    Braunstein, Mark L. (2022). FHIR. Computers in Health Care, 233–291. \url{https://doi.org/10.1007/978-3-030-91563-6_9}
    
    \bibitem[Cabarcos et al.(2019)]{Cabarcos2019}
    Cabarcos, P. A., Arias-Cabarcos, P., Krupitzer, C., \& Becker, C. (2019). A Survey on Adaptive Authentication. ACM Computing Surveys, 52(4), 80. \url{https://doi.org/10.1145/3336117}
    
    \bibitem[Doerfler et al.(2019)]{Doerfler2019}
    Doerfler, P., Thomas, K., Marincenko, M., Ranieri, J., Jiang, Y., Moscicki, A., \& McCoy, D. (2019). Evaluating Login Challenges as a Defense Against Account Takeover. 372–382. \url{https://doi.org/10.1145/3308558.3313481}
    
    \bibitem[Dutson et al.(2019)]{Dutson2019}
    Dutson, J., Allen, D., Eggett, D. L., \& Seamons, K. E. (2019). Don’t Punish all of us: Measuring User Attitudes about Two-Factor Authentication. 119–128. \url{https://doi.org/10.1109/eurospw.2019.00020}
    
    \bibitem[Feth dkk (2019)]{Feth2019}
    Feth, Denis, dan Svenja Polst. “Heuristics and Models for Evaluating the Usability of Security Measures.” Dalam Proceedings of Mensch und Computer 2019, 275–85. MuC ’19. New York, NY, USA: Association for Computing Machinery, 2019. doi:10.1145/3340764.3340789.
    
    \bibitem[Misbahuddin et al.(2017)]{Misbahuddin2017}
    Misbahuddin, M., B. S. Bindhumadhava, B. S. Bindhumadhava, Bindhumadhava, B. S., \& Dheeptha, B. (2017). Design of a risk-based authentication system using machine learning techniques. 1–6. \url{https://doi.org/10.1109/uic-atc.2017.8397628}
    
    \bibitem[Prasad et al.(2017)]{Prasad2017}
    Prasad, K. K., K, K. P., \& Aithal, S. (2017). A Study on Enhancing Mobile Banking Services Using Location Based Authentication. \url{https://doi.org/10.47992/ijmts.2581.6012.0006}
    
    \bibitem[Rahat et al.(2021)]{Rahat2021}
    Rahat, Tamjid Al, Feng, Yu, \& Tian, Yuan. (2021). Cerberus. Cornell University - ArXiv. \url{https://doi.org/10.1145/3548606.3559381}
    
    \bibitem[Roy \& Dasgupta(2018)]{Roy2018}
    Roy, A., \& Dasgupta, D. (2018). A fuzzy decision support system for multifactor authentication. Soft Computing - A Fusion of Foundations, Methodologies and Applications, 22(12), 3959–3981. \url{https://doi.org/10.1007/s00500-017-2607-6}
    
    \bibitem[Solapurkar(2016)]{Solapurkar2016}
    Solapurkar, P. (2016). Building secure healthcare services using OAuth 2.0 and JSON web token in IOT cloud scenario. International Conferences on Contemporary Computing and Informatics, 99–104. \url{https://doi.org/10.1109/ic3i.2016.7917942}
    
    \bibitem[Speiser et al.(2019)]{Speiser2019}
    Speiser, J. L., Miller, M., Miller, M. E., Tooze, J. A., \& Ip, E. H. (2019). A Comparison of Random Forest Variable Selection Methods for Classification Prediction Modeling. Expert Systems With Applications, 134, 93–101. \url{https://doi.org/10.1016/j.eswa.2019.05.028}
    
    \bibitem[Sujudi, Heryawan et al., 2022]{Sujudi2022}
    Sujudi, Hammam Mahfuzh, dan Lukman Heryawan. “An Automatic Data Mapping for Interoperability of OpenEMR Medical Practice Management Software Using the Fast Healthcare Interoperability Resources.” Advanced Biomedical Engineering 11 (2022): 186–93. doi:10.14326/abe.11.186.
    
    \bibitem[Taneja(2013)]{Taneja2013}
    Taneja, M. (2013). An analytics framework to detect compromised IoT devices using mobility behavior. Information and Communication Technology Convergence, 38–43. \url{https://doi.org/10.1109/ictc.2013.6675302}
    
    \bibitem[Thomas et al.(2017)]{Thomas2017}
    Thomas, K., Li, F., Zand, A., Barrett, J., Ranieri, J., Invernizzi, L., Markov, Y., Comanescu, O., Eranti, V., Moscicki, A., Margolis, D., Paxson, V., \& Bursztein, E. (2017). Data Breaches, Phishing, or Malware?: Understanding the Risks of Stolen Credentials. 1421–1434. \url{https://doi.org/10.1145/3133956.3134067}
    
    \bibitem[Wiefling et al.(2021)]{Wiefling2021}
    Wiefling, Stephan, Markus Dürmuth, \& Luigi Lo Iacono. (2021). What’s in Score for Website Users: A Data-driven Long-term Study on Risk-based Authentication Characteristics. Financial Cryptography. \url{https://doi.org/10.1007/978-3-662-64331-0_19}
    
    \bibitem[Wiefling et al.(2022)]{Wiefling2022}
    Wiefling, Stephan, Paul René Jørgensen, Sigurd Thunem, \& Luigi Lo Iacono. (2022). Pump Up Password Security! Evaluating and Enhancing Risk-Based Authentication on a Real-World Large-Scale Online Service. ACM Transactions on Privacy and Security. \url{https://doi.org/10.1145/3546069}
    
    \bibitem[Zhang et al.(2012)]{Zhang2012}
    Zhang, F., Kondoro, A., \& Muftic, S. (2012). Location-Based Authentication and Authorization Using Smart Phones. 2012 IEEE 11th International Conference on Trust, Security and Privacy in Computing and Communications, 1285–1292. \url{https://doi.org/10.1109/trustcom.2012.198}
\end{thebibliography}
