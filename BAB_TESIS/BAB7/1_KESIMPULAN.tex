Kesimpulan dari penelitian ini adalah sebagai berikut:
\begin{enumerate}
	\item Model yang dihasilkan belum dapat mengklasifikasi risiko autentikasi dengan baik. Sistem autentikasi M2M berbasis risiko menggunakan Random Forest dapat mengklasikasi risiko autentikasi. Dengan akurasi 70.8\%, presisi 70.1\%, \textit{recall} 96.8\%, dan \textit{F1-score} 71.3\%. Ketimpangan akurasi dan \textit{recall} disebabkan oleh ketidakseimbangan jumlah data pada kelas yang berbeda.
	\item Pembatasan fitur kepentingan dapat berpengaruh pada akurasi sistem.
\end{enumerate}
